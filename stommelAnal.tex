% ######################################################################
\begin{frame}
\frametitle{Bewegunsgleichungen}
%%....................................................................
\begin{align}
	\rho \Dpr{\vec{u}}{t} 
	=
	&- 2\vec{\Omega} \times \vec{u} -\;\;\;\;\nabla p  \;\;\;\;\;\;\;\;    + \nabla \cdot\boldsymbol{\mathsf{T}}\;\;\;\; +\;\;\; \rho\vec{g} \nonumber \\
	&-      Coriolis                -    Druckgrad.+             Reibung                 +   Gravi. 
\end{align}

%%....................................................................
\begin{align}
	\Dpr{m}{t}
	&=
	0
\end{align}
%%....................................................................

%%....................................................................
\end{frame}
% ######################################################################



% ######################################################################
\begin{frame}[noframenumber]
\frametitle{Approximationen/Manipulationen}
\end{frame}
\begin{frame}
\frametitle{Approximationen/Manipulationen}
\begin{itemize}
	\item<1-> 
	mesoskalige Turbulenz parametrisiert \\
	\textit{Reynolds averaging}
	\item<2-> 
	Corioliskraft linearisiert\\
	 $\vec{f}(y) = f_{0} + \beta y$
	\item<3->
	Hydrostatische Approximation \\
	$\rho\vec{g} = -\dpr{p}{z}$ %(nur noch horizontale Kräfte übrig)
	\item<4->
	zeitlich konstant \\
	$\dpr{\vec{u}}{t} = 0$
	\item<5->
	Inkompressibiltät \\
	Massenerhaltung $\rightarrow$ Volumenerhaltung ($\rho = const$)
	\item<6->
	Horizontale Geschwindigkeiten viel größer als vertikale \\
	$w=0$
	\item<7->
	vertikal integrieren \\
	$\int_{H} \vec{u} \dint{z}=	\vec{U}$
\end{itemize}
\end{frame}
% ######################################################################

\begin{frame}
	\frametitle{Viereckiges flaches Honigbecken}
	% TODO bild aus BSC atlantic -> bild AUS BSC viereck
\end{frame}

% ######################################################################
\begin{frame}
\frametitle{übrig bleibt...}
%%....................................................................
\begin{align}
	\rho \Dpr{\vec{u}}{t} 
	=
	&- 2\vec{\Omega} \times \vec{u} -\;\;\;\;\nabla p  \;\;\;\;\;\;\;\;    + \nabla \cdot\boldsymbol{\mathsf{T}}\;\;\;\; +\;\;\; \rho\vec{g} \nonumber 
\end{align}
%%....................................................................
\begin{center}
	$\Downarrow$
\end{center}
%%....................................................................
\begin{align}
	0
	=
	%&      -f\unitvec{z} \times \vec{U}\; -    gH\grad{h} \;\;\;\;\;\;\; + \mathbf{\tau}_{s}\;\;\;\;\; - \mathbf{\tau}_{b}\;\;\;\;\;\;\;\;\;\;\;\;\; \;\;\;\; + A\grad^{2} \vec{U}  \nonumber \\
	%&-      Coriolis                -    Druckgrad.  +  Wind  - Bodenreibung  + laterale\; Reibung          \nonumber
	&  -\l f_{0} + \beta y \r \unitvec{z} \times \vec{U} -    gH\grad{h}  + \mathbf{\tau}_{s} - \mathbf{\tau}_{b}+ A\grad^{2} \vec{U} 
\end{align}
%%....................................................................
\begin{align}
	\Dpr{\vec{U}}{t}
	&=
	0
\end{align}
%%....................................................................
%%....................................................................
\end{frame}
% ######################################################################

% ######################################################################
\begin{frame}
\frametitle{Curl Operation}
%%....................................................................
\begin{align}
	0
	=&  \grad \times \left[\;\;\;\;\;   -\vec{f} \times \vec{U}   -    g H \grad{h} + \mathbf{\tau}_{s} - \mathbf{\tau}_{b} + A  \grad^{2} \vec{U} \;\;\;\;\;   \right]  \nonumber \\
     &     \nonumber 
\end{align}
%%....................................................................
\end{frame}

\begin{frame}
\frametitle{Curl Operation}
%%....................................................................
\begin{align}
	0
	=&  \grad \times \left[ \;\;\;\;\;   -\vec{f}(y) \times \vec{U}   -    g H \grad{h} + \mathbf{\tau}_{s} - \mathbf{\tau}_{b} + A  \grad^{2} \vec{U}  \;\;\;\;\;   \right]  \nonumber \\
	=& \;\;\;\;\; \;\;\; -\beta V  + \vort{\l \mathbf{\tau}_{s}- \mathbf{\tau}_{b} \r}  + A  \grad \times \grad^{2} \vec{U}    \nonumber 
\end{align}
%%....................................................................
\end{frame}
% ######################################################################

% ######################################################################
\begin{frame}
\frametitle{Stromfunktion}
%%....................................................................
\begin{align}
 \vec{U}&=\tsca{\grad} \psi(x,y) 
\end{align}
%%...................................................................
soll heißen...
%%....................................................................
\begin{align}
	U &= -\dpr{\psi}{y}  \nonumber  \\
	V &= \;\;\; \dpr{\psi}{x}   \nonumber 
\end{align}
%%....................................................................
\end{frame}
% ######################################################################

% ######################################################################
\begin{frame}
\frametitle{Stromfunktion}
%%....................................................................
\begin{align}
		0
	=& -\beta V  + \vort{\l \mathbf{\tau}_{s}- \mathbf{\tau}_{b} \r}  + A  \grad \times \grad^{2} \vec{U}    \nonumber 
\end{align}
%%....................................................................
%\begin{center}
	%$\Downarrow$
%\end{center}
%%%....................................................................
%\begin{align}
		%0
	%=& -\beta \dpr{\psi}{x}  + \vort{\l \mathbf{\tau}_{s}- \mathbf{\tau}_{b} \r}  + A  \grad \times \grad^{2} \gradt \psi   \nonumber 
%\end{align}
%%%....................................................................
\begin{center}
	$\Downarrow$
\end{center}
%%....................................................................
\begin{align}
		0
	=& -\beta \dpr{\psi}{x}  + \vort{\l \mathbf{\tau}_{s}- \mathbf{\tau}_{b} \r}  + A  \grad^{4} \psi  
\end{align}
%%....................................................................
\end{frame}
% ######################################################################

% ######################################################################
\begin{frame}
\frametitle{weitere Approximationen}
\begin{itemize}
	\item<1-> 
	Bodenreibung proportional zu $\vec{U}$ \\
	$\vec{\tau_b} = R \vec{U}$
	\item<2->
	keine laterale Reibung! \\
	$A  \grad^{4} \psi  = 0 $
	\item<3->
	Windstress sinusoidale Funktion von $y$ \\
	$\vec{\tau_s} = F \cos{(\pi y / B)}$	
\end{itemize}
\end{frame}
% ######################################################################

% ######################################################################
\begin{frame}
\frametitle{weitere Approximationen}
%%....................................................................
\begin{align}
		0
	=& -\beta \dpr{\psi}{x}  + \vort{\l \mathbf{\tau}_{s}- \mathbf{\tau}_{b} \r}  + A  \grad^{4} \psi \nonumber  
\end{align}
%%....................................................................
\begin{center}
	$\Downarrow$
\end{center}
%%....................................................................
\begin{align}
		0
	=& -\beta \dpr{\psi}{x}  - R \vort{U} - F \grad \times  \cos{(\pi y / B)} \nonumber   \\
	%=& -\beta \dpr{\psi}{x}  - R \grad^2 \psi+ F \dpr{}{y} \cos{(\pi y / B)}  \nonumber  \\
	=& -\beta \dpr{\psi}{x}  - R \grad^2 \psi + \frac{F\pi}{B} \sin{(\pi y / B)}\nonumber  \\
R \grad^2 \psi+\beta \dpr{\psi}{x}   
	=&
	 F\alpha \sin{(\alpha y )}
\end{align}
%%....................................................................
...mit $ \alpha	= \frac{\pi}{B} $
\end{frame}
% ######################################################################

% ######################################################################
\begin{frame}
\frametitle{Analytische Lösung möglich}
Inhomogene partielle Differentialgleichung 2er Ordnung.
%%....................................................................
\begin{align}
    R \grad^2 \psi+\beta \dpr{\psi}{x}   
	=&
	 F\alpha \sin{(\alpha y )}\nonumber 
\end{align}
%%....................................................................
% TODO analy loesung zeigen
% TODO bild!
\end{frame}
% ######################################################################

% ######################################################################
\begin{frame}
\frametitle{Numerische Lösung}
\begin{itemize}
	\item
	variable Beckenform
	\item
	variabler Windstress
\end{itemize}
%%....................................................................
\begin{align}
    R \grad^2 \psi+\beta \dpr{\psi}{x}   
	=&
	 W(x,y)
\end{align}
%%....................................................................
% TODO bild!
\end{frame}
% ######################################################################

% ######################################################################
\begin{frame}
\frametitle{Jacobi-Methode}
Annahme: $\psi^{k=1}=rand$ überall
%%....................................................................
\begin{align}
    R \grad^2 \psi^k_{i,j} + \beta \dpr{}{x}\psi^k_{i,j} - W_{i,j}
    &=
    \Phi_{i,j}
\end{align}
%%....................................................................

%%....................................................................
\begin{align}
    R \grad^2 \psi^{k+1}_{i,j} + \beta \dpr{}{x} \psi^{k+1}_{i,j} - W_{i,j}
    &=
   0
\end{align}
%%....................................................................
\end{frame}
% ######################################################################





