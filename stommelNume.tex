
% ######################################################################
\begin{frame}
\frametitle{Numerische Lösung}
\begin{itemize}
	\item
	variable Beckenform
	\item
	variabler Windstress
\end{itemize}
% TODO bild!
\end{frame}
% ######################################################################


% ######################################################################
\begin{frame}
\frametitle{Finite Differenzen}
 TODO bild aus BSC gitter!
  \end{frame}
% ######################################################################

% ######################################################################
\begin{frame}
\frametitle{Finite Differenzen}
 Beispiel $\dpr{\psi(x)}{x}$ durch Taylorreihe diskretisiert:
 %%....................................................................
 \begin{align} \label{tayA}
	\psi(x + \delta x)
	&=
	\psi(x)
	+ \frac{\delta x}{1!} \dpr{\psi}{x}
	+ \frac{\delta x^2}{2!} \dpr{^2\psi}{x^2}
	+ ...
 \end{align}
 %%....................................................................
 %%....................................................................
 \begin{align} \label{tayB}
	\psi(x - \delta x)
	&=
	\psi(x)
	- \frac{\delta x}{1!} \dpr{\psi}{x}
	+ \frac{\delta x^2}{2!} \dpr{^2\psi}{x^2}
	+ ...
 \end{align}
 %%....................................................................
 \pause
 \eqref{tayA} - \eqref{tayB}:
 %%....................................................................
 \begin{align}
	\dpr{\psi}{x}
	&\approx
	\frac{\psi(x + \delta x) - 	\psi(x - \delta x)}{2 \delta x} \nonumber
 \end{align}
 %%.................................................................... 
 \end{frame}
% ######################################################################

\begin{frame}
	\frametitle{Höhere Ordnungen}
	Beispiel: $\grad^2$ an der Stelle $(0,0)$
	%%....................................................................
	\begin{align}
		&\grad^2 \psi(0,0)
		=
		 \dpr{^2 \psi}{x^2} + \dpr{^2 \psi}{ y^2} \nonumber\\
		&\approx
		\frac{\psi(1,0) - 2 \psi(0,0) + \psi(-1,0)}{\delta x^2}
		+
		\frac{\psi(0,1) - 2 \psi(0,0) + \psi(0,-1)}{\delta y^2}\nonumber
	\end{align}
	%%....................................................................
\end{frame}

% ######################################################################
\begin{frame}
\frametitle{Numerische Lösung}
\begin{align}
    R \grad^2 \psi  +  \beta \dpr{\psi}{x}\nonumber
	=&
	 W(x,y)
\end{align}
...erst mal ohne $\beta \dpr{\psi}{x}$
\pause
%%....................................................................
\begin{align}
    \grad^2 \psi
	=&
	 W(x,y)\nonumber
\end{align}
%%....................................................................
\pause
mit Randbedingung (willkürlich)
%%....................................................................
\begin{align}
	\psi_{bndry} = 0	\nonumber
\end{align}
%%....................................................................
% TODO bild!
\end{frame}
% ######################################################################


% ######################################################################
\begin{frame}
\frametitle{Jacobi-Methode}
\textit{first guess}: $\psi^{k=0}=random$ \\
%%....................................................................
\begin{align}
     \grad^2 \psi^k_{i,j} 
    &=
     W_{i,j} \nonumber
\end{align}
\pause
$k$:
\begin{align}
     \grad^2 \psi^k_{i,j}  - W_{i,j}
    &=
    \Phi_{i,j} \label{k0}
\end{align}
%%....................................................................
\pause
$k+1$:
%%....................................................................
\begin{align}
     \grad^2 \psi^{k+1}_{i,j}  - W_{i,j}
    &=
   0 \label{k1}
\end{align}
%%....................................................................
\end{frame}
% ######################################################################

% ######################################################################
\begin{frame}
\frametitle{Jacobi-Methode}
\eqref{k1} - \eqref{k0}:
%%....................................................................
\begin{align}
	\grad^2 \psi^{k+1}_{i,j} - \grad^2 \psi^{k}_{i,j}
    &=
    -\Phi_{i,j} 	
\end{align}
%%....................................................................
\end{frame}
% ######################################################################

% ######################################################################
\begin{frame}
\frametitle{Jacobi-Methode}
einsetzen :
%%....................................................................
\begin{align}
	\frac{\psi^{k}_{i+1,j} - 2 \psi^{k+1}_{i,j} + \psi^{k}_{i-1,j}}{\delta x^2}
-
	\frac{\psi^{k}_{i+1,j} - 2 \psi^{k  }_{i,j} + \psi^{k}_{i-1,j}}{\delta x^2}
+ \mathit{term_y}
    &=
    -\Phi_{i,j} 	\nonumber
\end{align}
\pause
\begin{align}
	   \psi^{k+1}_{i,j}      
    &=
	\frac{\delta x}{2}\Phi_{i,j} +  \psi^{k}_{i,j}  	\nonumber
\end{align}
%%....................................................................
\end{frame}
% ######################################################################
