%\begin{multicols}{2}


\begin{derivation}[Vorticity]\label{der:vort}
	With the identity
	%%--------------------------------------------------------------------
	\begin{equation}\begin{split}
	\vec{u} \cdot \grad \vec{u}
	&=
	\left( \grad \times  \vec{u}  \right)\times \vec{u}
	+ \grad \vert \vec{u}\vert ^{2}/2\\
	&=
	\omega \times \vec{u}
	+ \grad \vec{u}^{2}/2
	\end{split}\end{equation}
	\eqref{eq:NS2} becomes
	%%--------------------------------------------------------------------
	\begin{equation}\begin{split}
	\frac{\partial \vec{u}}{\partial t}
	+
	\grad \vert \vec{u}\vert ^{2}/2
	+
	\left( 2\vec{\Omega} + \vec{\omega} \right)  \times \vec{u}
	&=
	- \frac{1}{\rho} \vec{\nabla} p
	+
	\nu  \vec{\nabla}^{2} \vec{u}
	+
	\vec{g} \\
	\frac{\partial \vec{u}}{\partial t}
	+
	\grad \vert \vec{u}\vert ^{2}/2
	+
	\vec{\omega_{a}} \times \vec{u}
	&=
	- \frac{1}{\rho} \vec{\nabla} p
	+
	\nu  \vec{\nabla}^{2} \vec{u}
	+
	\vec{g}
	\end{split}\end{equation}
	%%--------------------------------------------------------------------
	Applying the curl operation to \eqref{eq:NS2} and assuming \eqref{eq:konti1}
	for	an incompressible fluid yields and equation for the vorticity
	%%--------------------------------------------------------------------
	\begin{equation}\begin{split}
	&\frac{\partial \vec{\omega}}{\partial t}
	+
	\curl \grad \vert \vec{u}\vert ^{2}/2
	+
	\curl \left( \omega_{a} \times \vec{u} \right) \\
	&=
	-\frac{1}{\rho} \curl \grad p
	-
	\grad \rho^{-1} \times \grad p
	+
	\nu \grad \times \grad^{2} \vec{u}
	+
	\grad \times \vec{g} \label{eq:vort1}
	\end{split}\end{equation}
	%%--------------------------------------------------------------------
	Annihilating all $\curl \vec{grad}$ and $\div \curl$ and making use of
	the
	identity
	%%--------------------------------------------------------------------
	\begin{equation}\begin{split}
	\grad \times \left( \vec{A} \times B \right)
	&=
	\vec{A} \left( \grad \cdot B \right)
	-
	B \left( \grad \cdot \vec{A} \right)
	+
	\left( B \cdot \grad \right) \vec{A}
	-
	\left( \vec{A} \cdot \grad \right) B
	\end{split}\end{equation}
	%%--------------------------------------------------------------------
	%%--------------------------------------------------------------------
	\eqref{eq:vort1} becomes
	%%--------------------------------------------------------------------
	\begin{equation}\begin{split}
	\frac{\partial \vec{\omega}}{\partial t}
	+
	\curl \left( \omega_{a} \times \vec{u} \right)
	&=
	-\frac{\grad \rho \times \grad p	}{\rho^{2}}
	+
	\nu \grad \times \grad^{2} \vec{u}\\
	%%--------------------------------------------------------------------
	\frac{\partial \vec{\omega}}{\partial t}
	+
	\left( \vec{\vec{u}} \cdot \grad \right) \vec{\vec{\omega}_{a}}
	-
	\vec{\vec{u}} \left( \grad \cdot \vec{\vec{\omega}_{a}} \right)
	&-
	\left( \vec{\vec{\omega}_{a}} \cdot \grad \right) \vec{\vec{u}}\\
	&=
	\B
	-
	\nu \curl \left( \curl \left( \curl \vec{u} \right) \right)\\
	%%--------------------------------------------------------------------
	\frac{D \vec{\omega_{a}}}{D t}
	&=
	\left( \vec{\vec{\omega}_{a}} \cdot \grad \right) \vec{\vec{u}}
	+
	\B
	-
	\nu \curl \left( \curl \omega \right)\\
	&=
	\left( \vec{\vec{\omega}_{a}} \cdot \grad \right) \vec{\vec{u}}
	+
	\B
	+
	\nu \grad^2 \vec{\omega}
	\end{split}\end{equation}
	Scaling considerations based on the small aspect ratio e.g. noting that
	$\B \sim \grad p \times \grad \rho$ is at first approximation
	limited to	the $x,y$ plane and that $U/H \gg W/L$ and assuming $\omega_z
	\gg	\omega_h $,	leads to:

	%%%....................................................................
	\begin{equation}\begin{split}
	\dpr{\vec{\omega} }{t}
	+
	\advec{\vec{\omega}}
	+
	\beta v \unitvec{z}
	&=
	\left(\omega_z +
	f\right) \dpr{ \vec{u}}{z}
	+
	\B
	%%--------------------------------------------------------------------
	\end{split}\end{equation}
	horizontal:
	\begin{equation}\begin{split}
	&\frac{1}{T} \frac{U}{H }
	+
	\left(\frac{U }{L} + \frac{W}{H} \right)
	\frac{U }{H}
	\sim
	\frac{U }{H} \frac{U}{L}
	+
	f \frac{U}{H}
	+
	\B \\
	\Rightarrow
	&\frac{U}{HT }
	+
	\frac{U^2 }{LH}
	+
	\frac{UW }{H^2}
	\sim
	\frac{U^2}{LH}
	+
	f \frac{U}{H}
	+
	\B
	%%--------------------------------------------------------------------
	\end{split}\end{equation}
	vertical:
	\begin{equation}\begin{split}
	&\frac{U}{LT}
	+
	\frac{U^2}{L^2}
	+
	\frac{W U}{ H L}
	+
	\beta V
	\sim
	\frac{UW}{LH}
	+
	f\frac{W}{H}
	\\
	\Rightarrow
	&\frac{U}{LT}
	+
	\frac{U^2}{L^2}
	+
	\beta V
	\sim
	\frac{WU}{HL}
	+
	f\frac{W}{H}
	\end{split}\end{equation}
	%%%....................................................................

	Hence at first order:
	\begin{equation}\begin{split}
	\Dpr{\vec{\omega}_h }{t}
	&=
	\left(f + \omega_z \right)\dpr{\vec{u}_h}{z}
	+
	\B
	\end{split}\end{equation}
	\begin{equation}\begin{split}
	\Dpr{\omega_z }{t}
	+
	\beta v
	&=
	\left(f + \omega_z \right)\dpr{w}{z}
	\end{split}\end{equation}
	%If we further assume quasi-geostrophic motion so that any change in
	$\vec{\omega}_h$ is due to small ageostrophic parallelization of $\grad p$
	and	$\grad \rho$  via $B$ the tilting terms vanish, since then
	$\vec{\omega}_h$	normal to the plane spanned by $\grad \vec{u}_h$ and
	$\grad	w$
\end{derivation}

%%####################################################################
%%####################################################################


\begin{derivation}[Kinetic Energy]\label{der:Ek}
Multiply \eqref{eq:NS2} by $ \vec{u}$:
\begin{equation}\begin{split}
\vec{u} \cdot{}
\left(
\frac{\partial \vec{u}}{\partial t}
+
\vec{u}\cdot \grad \vec{u}
+
2\vec{\Omega} \times \vec{u}
\right)
&=
-
\vec{u} \cdot \frac{1}{\rho} \vec{\nabla} p
+
\vec{u}\cdot \nu  \vec{\nabla}^{2} \vec{u}
+
\vec{u}\cdot  \vec{g}\\
%%--------------------------------------------------------------------
\frac{1}{2}\frac{\partial \vec{u}^{2}}{\partial t}
+
\frac{1}{2}  \vec{u}\cdot \grad \vec{u}^{2}
+
\vec{u} \cdot 2\vec{\Omega} \times \vec{u}
&=
-
\vec{u} \cdot \frac{1}{\rho} \vec{\nabla} p
+
\vec{u}\cdot \nu  \vec{\nabla}^{2} \vec{u}
-
w g \\
%%--------------------------------------------------------------------
\frac{1}{2}\frac{\partial \vec{u}^{2}}{\partial t}
+
\frac{1}{2}  \vec{u}\cdot \grad \vec{u}^{2}
&=
-
\vec{u}_{h} \cdot \frac{1}{\rho} \grad_{h} p
+
w g
+
\vec{u}\cdot \nu  \vec{\nabla}^{2} \vec{u}
-
w g \\
%%--------------------------------------------------------------------
\frac{1}{2}\frac{\partial \vec{u}^{2}}{\partial t}
+
\frac{1}{2}  \vec{u}\cdot \grad \vec{u}^{2}
&=
-
\vec{u}_{h} \cdot \frac{1}{\rho} \grad_{h} p
+
\vec{u}\cdot \nu  \vec{\nabla}^{2} \vec{u} \\
%%--------------------------------------------------------------------
\frac{\partial \Ek}{\partial t}
+
\vec{u}\cdot \grad \Ek
&=
-
g \vec{u}_{h} \cdot  \grad \eta(x,y)
+
\nu\left(
\frac{1}{2}\grad^2 \vec{u}^2 - \norm{\grad \vec{u}}^2
\right)
\end{split}\end{equation}
\end{derivation}

%%####################################################################
%%####################################################################

\begin{derivation}[Mechanical Energy]
\label{der:Em}
Add term for potential energy to \eqref{eq:Ekin1} (assuming $\grad \rho=0$)
\begin{equation}\begin{split}
\frac{D E_{m}}{D t}
&=
-
g \vec{u} \cdot  \grad \eta(x,y)
+
\nu\left(
\frac{1}{2}\grad^2 \vec{u}^2 - \norm{\grad \vec{u}}^2
\right)
+
\vec{u}\cdot\vec{g}\\
%%--------------------------------------------------------------------
\frac{D E_{m}}{D t}
&=
-
g \vec{u} \cdot  \grad \eta(x,y)
+
\nu\left(
\frac{1}{2}\grad^2 \vec{u}^2 - \norm{\grad \vec{u}}^2
\right)
-wg\\
%%--------------------------------------------------------------------
&=
-
g \left( \frac{\partial \eta}{\partial t}
+
\vec{u} \cdot  \grad \eta \right)
+
\nu\left(
\frac{1}{2}\grad^2 \vec{u}^2 - \norm{\grad \vec{u}}^2
\right)\\
%%--------------------------------------------------------------------
&=
-
g \frac{D \eta}{D t}
+
\nu\left(
\frac{1}{2}\grad^2 \vec{u}^2 - \norm{\grad \vec{u}}^2
\right)\\
%%--------------------------------------------------------------------
&=
\nu\left(
\frac{1}{2}\grad^2 \vec{u}^2 - \norm{\grad \vec{u}}^2
\right)
\end{split}\end{equation}
\end{derivation}


%%####################################################################
%%####################################################################

\begin{derivation}[Enstrophy]
\label{der:enstro}
In 2 dimensions the definition of enstrophy can also be rewritten as:
%%%....................................................................
\begin{equation}\begin{split}
\Enstro
=
\inta{\enstro}
&=\inta{
\norm{\grad \vec{u}}^2}
\\
&=\inta{
\left(\frac{\partial v}{\partial x}\right)^2
+ \left(\frac{\partial u}{\partial y} \right)^2
+\left(\frac{\partial v}{\partial y}\right)^2
+ \left(\frac{\partial u}{\partial x} \right)^2}
\\
&=\inta{
\left(\frac{\partial v}{\partial x}\right)^2
+ \left(\frac{\partial u}{\partial y} \right)^2
+\left(\frac{\partial v}{\partial y}\right)^2
+ \left(\frac{\partial u}{\partial x} \right)^2
-\left(\div \vec{u}\right)^{2}}
\\
&=\inta{
\left(\frac{\partial v}{\partial x}\right)^2
+ \left(\frac{\partial u}{\partial y} \right)^2
-2 \dpr{u}{x}\dpr{v}{y}
}
\\
&=\inta{
\omega^{2}
+2 \dpr{u}{y}\dpr{v}{x}
-2 \dpr{u}{x}\dpr{v}{y}
}\\
&=\inta{
\omega^{2}
+2 \dpr{u}{y}\dpr{v}{x}
+2 \left(\dpr{v}{y} \right)^2
}
\end{split}\end{equation}
with $\div \vec{u}=0$ and appropriate boundary conditions, the last two
terms	cancel in the integral leaving

\begin{equation}\begin{split}
\Enstro
&=\inta{
\omega^{2}
}
\end{split}\end{equation}

\end{derivation}

%%####################################################################
%%####################################################################

\begin{derivation}[Vorticity Scales]
\label{der:vortscale}
Assuming approximate geostrophy and $\Bu = \order{1}$:
%%%....................................................................
\begin{equation}\begin{split}
\rho fU
&\sim
\grad p
\end{split}\end{equation}
%%%....................................................................
in a layered model:
%%%....................................................................
\begin{equation}\begin{split}
fU
&\sim
g' \grad \eta \\
U
&\sim
\frac{h N^{2}}{f} \grad \eta \\
U
&\sim
\frac{h^{2} N^{2}}{f\mathrm{L}}
\end{split}\end{equation}
%%%....................................................................
and hence
%%%....................................................................
\begin{equation}\begin{split}
\vec{\omega}
\sim
U/\mathrm{L}
&\sim
\frac{h^{2} N^{2}}{f\mathrm{L}^{2}} \\
&=
\frac{h^{2} N^{2}}{f\mathrm{\Lr}^{2}} \frac{\Lr}{L}\\
&=
f\frac{\Lr}{L}
\end{split}\end{equation}
%%%....................................................................
\end{derivation}
%###################################%\\%###################################%

\begin{derivation}[vortex limitations]
%%%....................................................................
\begin{equation}\begin{split}
\advec{\vec{u}}
+
f \tvec{u}
&=
-g \grad \eta
\end{split}\end{equation}
%%%....................................................................
\begin{equation}\begin{split}
\frac{U^{2}}{L}
+
f U
+
\frac{gh}{L}
&=
0 \\
U^{2}
+
f LU
+
gh
&=
0 \\
\end{split}\end{equation}
%%%....................................................................
possible balances:
%%%....................................................................
cyclone\\$\sign{F_{c}}=\sign{F_{p}}$:
\begin{equation}\begin{split}
U_{1,2}
&=
-fL/2
\pm
\sqrt
{
f^{2}L^{2}/4
+
gh
}
\end{split}\end{equation}
%%%....................................................................
%%%....................................................................
anti-cyclone\\
$\sign{F_{c}}=-\sign{F_{p}}$:
\begin{equation}\begin{split}
U_{1,2}
&=
-fL/2
\pm
\sqrt
{
f^{2}L^{2}/4
-
gh
}
\end{split}\end{equation}

%%%....................................................................
\begin{equation}\begin{split}
4g h
&\le
f^{2}L^{2}\\
2c
&\le
fL\\
2 \mathrm{L_{R}}
&\le
L\\
\end{split}\end{equation}
%%%....................................................................
\end{derivation}

%%####################################################################
%%####################################################################

\begin{derivation}[Okubo-Weiss-Parameter]\label{der:okubo}
%%%....................................................................
\begin{equation}\begin{split}
\ten{T}
&=
\grad \vec{u}\\
%%--------------------------------------------------------------------
&=
\dpr{u_i}{x_j} \unitvec{e}_i \unitvec{e}_j \\
&=
\frac{1}{2}\left(
\left(\ten{T}+\tent{T}\right)
+
\left( \ten{T} - \tent{T} \right)
\right) \\
%%--------------------------------------------------------------------
&=
\frac{1}{2} \left( \left( \dpr{u_i}{x_j} + \dpr{u_j}{x_i}\right)
\unitvec{e}_i	\unitvec{e}_j \right)
+
\frac{1}{2} \left( \left( \dpr{u_i}{x_j} - \dpr{u_j}{x_i}\right)
\unitvec{e}_i	\unitvec{e}_j \right)
\end{split}\end{equation}
%%%....................................................................
%%%....................................................................
\begin{equation}\begin{split}
\mathrm{O_w}=\tr{\ten{T}^2}
&=
\left(
\dpr{u_i}{x_k} \dpr{u_k}{x_j} \unitvec{e}_i \unitvec{e}_j
\right)_{i,i}\\
&=
\dpr{u_i}{x_j} \dpr{u_j}{x_i}\\
%%--------------------------------------------------------------------
&=
\left(\dpr{u_i}{x_i}\right)^2
+\left(1-\delta_{i,j}\right)\dpr{u_j}{x_i} \dpr{u_i}{x_j} \\
%%--------------------------------------------------------------------
&=
\left(\dpr{u_i}{x_i}\right)^2
+
\frac{1}{2}\left( \dpr{u_i}{x_j} + \dpr{u_j}{x_i} \right)^2
-\frac{1}{2}\left( \dpr{u_i}{x_j} - \dpr{u_j}{x_i} \right)^2 \\
%%--------------------------------------------------------------------
&=
\frac{1}{2}\left( \dpr{u_i}{x_i} + \dpr{u_j}{x_j} \right)^2
+\frac{1}{2}\left( \dpr{u_i}{x_i} - \dpr{u_j}{x_j} \right)^2
+
\frac{1}{2}\left( \dpr{u_i}{x_j} + \dpr{u_j}{x_i} \right)^2
-\frac{1}{2}\left( \dpr{u_i}{x_j} - \dpr{u_j}{x_i} \right)^2 \\
%%--------------------------------------------------------------------
&=
divergence^2
+ stretching^2
+ shear^2
- vorticity^2
\end{split}\end{equation}
Hence for motion dominated by deformation and shear the system has
hyperbolic	character, whereas vorticity-dominated motion has parabolic
character. Of interest should therefor not only be the value of $\mathrm{O_w}$
but also its	gradient. An abrupt change in $\mathrm{O_w}$ clearly identifies
regions of	vorticity genesis and decay.
\\ In 2 dimensions:
%%%....................................................................
\begin{equation}\begin{split}
\okubo
&=
\left(\dpr{u}{x}\right)^2
+
2 \dpr{u}{y} \dpr{v}{x}
\end{split}\end{equation}
%%%....................................................................
\end{derivation}

%%####################################################################
%%####################################################################
\begin{derivation}[Cushman's Drift Speed]

%%%....................................................................
\begin{equation}\begin{split}
X_t
&=
-\frac{\beta g'}{f_0^2} \frac{\int H\eta + \eta^2/2 \; \mathrm{d}A}{\int\eta \; \mathrm{d}A}\\
&=
-\frac{\beta g'}{f_0^2} \left(H +\frac{ \int  \eta^2/2 \; \mathrm{d}A}{\mathrm{V_e}} \right) \\
&=
-\frac{\beta c^2}{f_0^2} \left(1 +\frac{1}{H} \frac{ \int  \eta^2/2 \;\mathrm{d}A	}{\mathrm{V_e}} \right) \\
&=
-\beta \Lr^2 \left(1 +\frac{1}{H} \frac{ \int  \eta^2 \; \mathrm{d}A}{2\mathrm{V_e}} \right) \\
&=
\frac{\omega_{long}}{k} \left(1 +\frac{1}{H} \frac{ \int  \eta^2 \;\mathrm{d}A}{2\mathrm{V_e}} \right) \\
\end{split}\end{equation}
%%%....................................................................
\end{derivation}

%###################################%\\%###################################%
\begin{derivation}[Mean Fields of $u$ and $b$]
\label{der:fields}
%%%................................................................
\begin{subequations}\label{eq:inhomo1}
\begin{align}
\dpr{u_i}{t} \unitvec{e}_i
+
u_j \dpr{u_i}{x_j}\unitvec{e}_i
+
\delta_{j3}\timesES{f}{u}
&=
-\dpr{p}{x_i}	\unitvec{e}_i
+ \nu \dpr{^2u_i}{x_j^2}\unitvec{e}_i
- \delta_3 g\unitvec{e}_i \label{eq:inhomoNS}\\
% --------------------------------------------------------------------
\dpr{u_i}{x_i}
&=
0 \label{eq:inhomoCON}\\
% --------------------------------------------------------------------
\Dpr{\rho}{t}
&=
-\dpr{J_{i}^{rad}}{x_i}
+   \kappa \dpr{^2 \rho}{x_i} \label{eq:inhomoSTATE}
\end{align}
\end{subequations}
% %%....................................................................
A Reynolds decomposition of \eqsref{eq:inhomo1} yields
% %%....................................................................
\begin{subequations} \label{eq:inhomoDECOMP}
\begin{align}
\begin{split}
\dpr{\inbr{\decom{u_i}}}{t}\unitvec{e}_i
+
\inbr{\decom{u_j}} \dpr{\inbr{\decom{u_i}}}{x_j}\unitvec{e}_i
+
\delta_{j3}\timesES{f}{\inbr{\decom{u}}}\\
=
-\dpr{\inbr{\decom{p}}}{x_i}	\unitvec{e}_i
+ \nu \dpr{^2\inbr{\decom{u_i}}}{x_j^2}\unitvec{e}_i
- \delta_3 \ol{g} \unitvec{e}_i\label{eq:decompfin}
&\end{split}\\
% --------------------------------------------------------------------
\dpr{\inbr{\decom{u_i}}}{x_i}
=
0 & \\
% --------------------------------------------------------------------
\dpr{\inbr{\decom{\rho}}}{t}
+	\inbr{\decom{u_i}}\dpr{\inbr{\decom{\rho}}}{x_i}
=
-	\ol{\dpr{J_{i}^{rad}}{x_i}}
-	\dpr{J_{i}^{rad}}{x_i}'
+   \kappa \dpr{^2 \inbr{\decom{\rho}}}{x_i}
\end{align}
\end{subequations}
% %%....................................................................
Averaging \eqsref{eq:inhomoDECOMP} yields
% %%....................................................................
\begin{subequations}
\begin{align}
\begin{split}
&\dpr{\ol{\inbr{\decom{u_i}}}}{t}  \unitvec{e}_i
+  \\
&\ol{\ol{u_j} \dpr{\ol{u_i}}{x_j}\unitvec{e}_i
+
\delta_{j3}\timesES{f}{\inbr{\decom{u}}}
+
\ol{u_j} u_i'\unitvec{e}_i
+
u_j'\dpr{\ol{u_i}}{x_j}\unitvec{e}_i
+
u_j'\dpr{u_i'}{x_j}\unitvec{e}_i
}\\
&=
-\ol{\dpr{\inbr{\decom{p}}}{x_i}}\unitvec{e}_i
+ \nu \dpr{^2\ol{\inbr{\decom{u_i}}}}{x_j^2}\unitvec{e}_i
- \delta_3 \ol{g}\unitvec{e}_i
\end{split} \\
% --------------------------------------------------------------------
& \dpr{\ol{\inbr{\decom{u_i}}}}{x_i}
=
0 \\
% --------------------------------------------------------------------
&\dpr{\ol{\inbr{\decom{\rho}}}}{t}
+\ol{
\ol{u_j} \dpr{\ol{\rho}}{x_j}
+
\ol{u_j} \rho'
+
u_j'\dpr{\ol{\rho}}{x_j}
+
u_j'\dpr{\rho'}{x_j}
}
=
-	\ol{\ol{\dpr{J_{i}^{rad}}{x_i}}-	\dpr{J_{i}^{rad}}{x_i}'}
+   \kappa \dpr{^2 \ol{\inbr{\decom{\rho}}}}{x_i}
\end{align}
\end{subequations}
% %....................................................................
with $\ol{\ol{\inbr{\;}}}=\ol{\inbr{\;}}$ and $\ol{\inbr{'}}=0$
% %....................................................................
\begin{subequations}
\begin{align}
\dpr{\ol{u_i}}{t}\unitvec{e}_i
+
\ol{\ol{u_j} \dpr{\ol{u_i}}{x_j}}\unitvec{e}_i
+
\ol{u_j'\dpr{u_i'}{x_j}}\unitvec{e}_i
+
\delta_{j3}\timesES{f}{\ol{u}}
&=
-\dpr{\ol{p}}{x_i}	\unitvec{e}_i
+ \nu \dpr{^2\ol{u_i}}{x_j^2}\unitvec{e}_i
- \delta_3 \ol{g}\unitvec{e}_i \\
% --------------------------------------------------------------------
\dpr{\ol{u_i}}{x_i}
&=
0 \\
% --------------------------------------------------------------------
\dpr{\ol{\rho}}{t}
+\ol{\ol{u_j} \dpr{\ol{\rho}}{x_j}}
+\ol{u_j'\dpr{\rho'}{x_j}}
&=
-\ol{\dpr{J_{i}^{rad}}{x_i}}
+   \kappa \dpr{^2 \ol{\rho}}{x_i^2}
\end{align}
\end{subequations}
% %....................................................................
Focusing solely on deviations from the mean state and assuming horizontal
gradients of the mean state to be negligible and hence due to continuity
also	\footnote{$\delta_h=1-\delta_3$} $\dpr{w}{z}=0 \rightarrow w=0$:

% %....................................................................
\begin{subequations}
\begin{align}
\delta_h \dpr{\ol{u_i}}{t}\unitvec{e}_i
+
\ol{w'\dpr{u_i'}{z}}\unitvec{e}_i
+
\delta_h \delta_{j3}\timesES{f}{\ol{u}}
&=
-\dpr{\ol{p}}{z}	\unitvec{e}_z
-  g\unitvec{e}_z \\
% --------------------------------------------------------------------
\dpr{\ol{\rho}}{t}
+\ol{w'\dpr{\rho'}{z}}
&=
-\dpr{\ol{J_{i}^{rad}}}{z}
+   \kappa \dpr{^2 \ol{\rho}}{z^2}
\end{align}
\end{subequations}
%....................................................................
returning to vector notation
% %....................................................................
\begin{subequations}
\begin{align}
\dpr{\ol{\vec{u}_h}}{t}
+
\ol{w' \dpr{\vec{u}'}{z}}
+
f \vec{k} \times \ol{\vec{u}_h}
&=
-\dpr{\ol{p}}{z} \unitvec{e}_z
+  \vec{g} \label{eq:inhomoNStemp1} \\
% --------------------------------------------------------------------
\dpr{\ol{\rho}}{t}
+\ol{w' \dpr{ \rho'}{z}}
&=
-\dpr{\ol{\vec{J}}_{rad}}{z}
+   \kappa \dpr{^2 \ol{\rho}}{z^2}
\end{align}
\end{subequations}
% %....................................................................
where the RHS of \eqref{eq:inhomoNStemp1} is hydrostaticity, leaving
$\ol{w'\dpr{w'}{z}}=0 \rightarrow \dpr{w'}{z}=0$ for the vertical part,
hence:
% %....................................................................
\begin{subequations}
\begin{align}
\dpr{\ol{\vec{u}_h}}{t}
+
\ol{ \dpr{w'\vec{u}_h'}{z}}
+
f \vec{k} \times \ol{\vec{u}_h}
&=
0 \\
% --------------------------------------------------------------------
\dpr{\ol{\rho}}{t}
+\ol{ \dpr{ w'\rho'}{z}}
&=
-\dpr{\ol{\vec{J}}_{rad}}{z}
+   \kappa \dpr{^2 \ol{\rho}}{z^2}
\end{align}
\end{subequations}
% %....................................................................
% %....................................................................
\begin{subequations} \label{eq:inhomo_fin}
\begin{align}
\dpr{\ol{\vec{u}_h}}{t}
+
\ol{ \dpr{w'\vec{u}_h'}{z}}
+
f \vec{k} \times \ol{\vec{u}_h}
&=
0 \\
% --------------------------------------------------------------------
\dpr{\ol{\rho}}{t}
&=
-\dpr{}{z}
\left(
\ol{  w'\rho'}
-\ol{\vec{J}}_{rad}
+   \kappa \dpr{ \ol{\rho}}{z} \right)
\end{align}
\end{subequations}
% %....................................................................
\end{derivation}
\todo[color=red]{diffusion term very small on macro scale?}
%%####################################################################
%%####################################################################
\begin{derivation}[Turbulent Kinetic Energy]
\label{der:turbkin}
Multiplication of \eqref{eq:decompfin}  with $u_i'$ %, averaging and summing
over	all $i$ yields
% %....................................................................
\begin{align} \notag
\begin{split}
u_i'\dpr{\inbr{\decom{u_i}}}{t}
+
u_i'\inbr{\decom{u_j}} \dpr{\inbr{\decom{u_i}}}{x_j}
+
u_i'\delta_{j3}\epsilon_{jki} f_j \inbr{\decom{u}}_k\\
=
-
u_i'\dpr{\inbr{\decom{p}}}{x_i}
+
u_i' \nu \dpr{^2\inbr{\decom{u_i}}}{x_j^2}
-
u_i' \delta_3 \ol{g}
&\end{split}
\end{align}
% %....................................................................
\begin{align}
\begin{split}
\oh \dpr{u'^2}{t}
+
\oh \ol{u_j} \dpr{u_i'^2}{x_j}
+
u_i'\ol{u_j} \dpr{\ol{u_i}}{x_j}
+
u_i'u_j' \dpr{\ol{u_i}}{x_j}
+
\oh  u_j' \dpr{u_i'^2}{x_j}\\
=
-
u_i'\dpr{\ol{p}}{x_i}
-
u_i'\dpr{p'}{x_i}
+
u_i' \nu \dpr{^2\ol{u_i}}{x_j^2}
+
\oh \nu \dpr{^2u_i'^2}{x_j^2}
-
\nu \left(\dpr{u_i'}{x_j}\right)^2
-
u_i' \delta_3  g
&\end{split}
\end{align}
% %....................................................................
First and last term on RHS are again hydrostaticity
% %....................................................................
\begin{align}
\begin{split}
\oh \dpr{u'^2}{t}
+
\oh \ol{u_j} \dpr{u_i'^2}{x_j}
+
u_i'\ol{u_j} \dpr{\ol{u_i}}{x_j}
+
u_i'u_j' \dpr{\ol{u_i}}{x_j}
+
\oh  u_j' \dpr{u_i'^2}{x_j}\\
=
-
u_i'\dpr{p'}{x_i}
+
u_i' \nu \dpr{^2\ol{u_i}}{x_j^2}
+
\oh \nu \dpr{^2u_i'^2}{x_j^2}
-
\nu \left(\dpr{u_i'}{x_j}\right)^2
&\end{split}
\end{align}
% %....................................................................
averaging...
% %....................................................................
\begin{align}
\begin{split}
\dpr{E_t}{t}
+
\ol{u_j} \dpr{E_t}{x_j}
-
\nu \dpr{^2 E_t}{x_j^2}
&=
-
\ol{u_i'\ol{u_j} \dpr{\ol{u_i}}{x_j}}
-
\ol{ u_i'u_j' \dpr{\ol{u_i}}{x_j}}
-
\oh \ol{ u_j' \dpr{u_i'^2}{x_j}}
-
\ol{ u_i'\dpr{p'}{x_i}}
+
\nu \ol{ u_i'\dpr{^2\ol{u_i}}{x_j^2}}
-
\nu \ol{\left(\dpr{u_i'}{x_j}\right)^2	}
\end{split}
\end{align}
\newline
\ldots
\newline
\begin{align}
\dpr{E_t}{t}
+
\ol{u_j} \dpr{E_t}{x_j}
&=
\dpr{}{x_j}
\left(
\nu \dpr{ E_t}{x_j}
+
\ol{u_j' p'}
+
\oh \ol{u_j' u_i' u_i'}
\right)
-
\ol{u_j' u_i'} \dpr{\ol{u}_i}{x_j}
+
\ol{b'w'}
-
\nu \ol{\left(\dpr{u_i'}{x_j}\right)^2	}  \label{eq:TurbKinFin}
\end{align}
\ldots
% %....................................................................
\begin{align}
\dpr{E_t}{t}
+
\ol{u_j} \dpr{E_t}{x_j}
&=
\dpr{\psi_i}{x_j}
-
\ol{u_j' u_i'} \dpr{\ol{u}_i}{x_j}
+
\ol{b'w'}
-
\nu \ol{\left(\dpr{u_i'}{x_j}\right)^2	} \label{eq:TurbKinFin2}
\end{align}
\ldots
% %....................................................................
with $\vec{\psi}=\nu \grad E_t +\ol{\vec{u}' p'} + \oh \ol{\vec{u}'
\vec{u}'^2}	$	as the total flux of turbulent kinetic energy.\\
Invoking again horizontal homogeneity as was done for \eqref{eq:inhomoNStemp1},
\eqref{eq:TurbKinFin2} takes the form
\begin{align}
\dpr{E_t}{t}
+
\ol{w} \dpr{E_t}{z}
&=
\dpr{\vec{\psi}}{z}
-
\ol{ \vec{u}_h' w'\dpr{\ol{\vec{u}}_h}{z}}
+
\ol{b'w'}
-
\nu \ol{  \left(\grad \vec{u}' \right)^2	}
\label{eq:TurbKinFin_horhomo}
\end{align}
\end{derivation}
\todo[color=red]{derivation still incomplete.. i assume
$\ol{\ol{a}b'}=\ol{a}\ol{b}$ might help..?}
